\documentclass[12pt,letterpaper]{article}
\usepackage{fullpage}
\usepackage[top=2cm, bottom=4.5cm, left=2.5cm, right=2.5cm]{geometry}
\usepackage{lastpage}
\usepackage{enumerate}
\usepackage{fancyhdr}
\usepackage[utf8]{inputenc}
\usepackage{color}
\usepackage[usenames,dvipsnames]{xcolor}
\usepackage{listings}


% Some smart developer wrote a Julia display style for us to use!
% TODO: You have to add jlcode.sty to your directory to make this work
% Download it here: https://github.com/wg030/jlcode
% Set usebox=true to get code in a box
\usepackage[autoload=false,usebox=false]{jlcode} % Use jlcode but don't let it override other listing styles
% For another way to display Julia code (that probably doesn't work with Unicode characters), see: https://tex.stackexchange.com/questions/212793/how-can-i-typeset-julia-code-with-the-listings-package


% General style options, mostly controlling line numbers
\lstset{
	frame            = lines,
	showstringspaces = false,
	numbers          = left,          % Line numbers left
	numberstyle      = \color{black}, % Line numbers are black, not deepblue
	breaklines       = true,          % Wrap overly-long lines
}

% Some custom colors if we want them
%\definecolor{deepblue}{rgb}{0,0,0.5}
\definecolor{deepred}{rgb}{0.6,0,0}
%\definecolor{deepgreen}{rgb}{0,0.5,0}



% Style for Matlab
\lstdefinestyle{Matlab}{
	language        = Matlab,
	frame           = lines, 
	basicstyle      = \ttfamily\footnotesize,
	keywordstyle    = \color{blue},
	stringstyle     = \color{deepgreen},
	commentstyle    = \color{deepred},
}



% Python style from https://tex.stackexchange.com/questions/83882/how-to-highlight-python-syntax-in-latex-listings-lstinputlistings-command/83883#83883

% Python style for highlighting
\lstdefinestyle{Python}{
		language         = Python,
		basicstyle       = \ttfamily\footnotesize,     % Font style (fixed width) and size
		otherkeywords    = {self,True,False},     % Add keywords you want to take keywordstyle here
		keywordstyle     = \bfseries\color{blue},
		emph             = {__init__,__name__,},  % Custom highlighting, add important things
		emphstyle        = \color{red}\bfseries,       % Custom highlighting style
		stringstyle      = \color{deepgreen}, % Color of "strings"
		commentstyle     = \color{deepred},   % Color of #comments
}



% TODO: Edit these as appropriate
\newcommand\course{AA 222}
\newcommand\hwnumber{5}
\newcommand\Name{Your name}
\newcommand\NetID{Your netID}

\pagestyle{fancyplain}
\headheight 35pt
\lhead{\NetID\\ \Name}
\chead{\textbf{\large Code Listing Example}}
\rhead{\course \\ \today}
\lfoot{}
\cfoot{\thepage\ of \pageref{LastPage}}
\rfoot{}
\headsep 1.5em



\begin{document}

	% TODO: Learn to print code from a file (make sure the filepath is correct)
%	\lstset{caption={Description of your code!}}
%	\lstinputlisting[style=Python]{./path/to/code.py}

	% TODO: Embed formatted equations in comments using mathescape=true
	%\lstinputlisting[style=Python,mathescape=true]{hw5_code/problem_1/part_a.py}
	

	
% SAMPLES
% Remember that you DON'T usually want to paste code directly into your document.
% It's much more error-proof to input the code directly from the code files.
% This is just an example.
	
	
% TODO: Learn to add Python code
% An example to show this in action, because it's really cool:
\begin{lstlisting}[caption={Python example. Adding equations in comments. Optimal control example.},style=Python,mathescape=true]
# Some code that does very little, but has embedded math in comments
# Linear system: get $x_{t+1} = Ax_t + Bu_t$.
xp = A @ x + B @ u
# Compute quadratic cost $c(x,u) = x^TQx + u^TRu$.
c = x.T @ Q @ x + u.T @ R @ u
\end{lstlisting}


% TODO: Learn to add Julia code
\begin{lstlisting}[caption={Julia example. Cross-entropy method from Prof. Kochenderfer's textbook. Note that most Unicode characters will work.},language=julia, style=jlcodestyle,mathescape=true]
# The cross-entropy method from "Algorithms for Optimization"
using Distributions
function cross_entropy_method(f,P,k_max,m=100,m_elite=10)
	for k ∈ 1:k_max
		samples = rand(P,m)
		order = sortperm([f(samples[:,i])for i ∈ 1:m]) 
		P = fit(typeof(P), samples[:, order[1:m_elite]])
	end
	return P
	@macro
end
\end{lstlisting}


% TODO: Learn to add Matlab code
\begin{lstlisting}[caption={Matlab example.},language=matlab, style=matlab,mathescape=true]
x = linspace(0, 10);
y = sin(x); % Plot $y = \sin(x)$.
plot(x,y);
\end{lstlisting}


	

% TODO: If you want to include some code output:
	% \begin{verbatim}
	% paste output here and it will be unformatted fixed-width
	% \end{verbatim}


\end{document}